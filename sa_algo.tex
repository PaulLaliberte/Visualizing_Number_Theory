\documentclass[9pt]{article}
\usepackage{amsmath}
\usepackage{hyperref}
\usepackage{amsthm}
\usepackage{mathtools}
\usepackage{amsfonts}
\usepackage[mathscr]{euscript}
\usepackage{color}
\usepackage{parskip}
\usepackage{algorithm}
\usepackage[noend]{algpseudocode}
\usepackage{xcolor}
\usepackage{commath}
\usepackage{ textcomp }
\usepackage{titling}
\usepackage[margin=2cm]{geometry}

\newcommand{\ssitem}[1][black]{\stepcounter{enumii}\item[\color{#1}$\bm{*}$\,\textbf{\alph{enumii})}]}
\newcommand{\nat}{\mathbb{N}}
\newcommand{\inte}{\mathbb{Z}}
\newcommand{\comp}{\mathbb{C}}
\newcommand{\lton}{\displaystyle \lim_{n \to \infty}}
\newcommand{\ep}{\epsilon}
\newcommand{\reals}{\mathbb{R}}
\newcommand{\ra}{\textcolor{red} * \textcolor{black}}
\setlength\parindent{0pt}

\makeatletter
\renewcommand*{\ALG@name}{Function}
\makeatother


\begin{document}

\Large
\textbf{Preliminaries}
\normalsize
\begin{itemize}
	\item $\mathcal{O}_K$: The ring of integers.
	\item ect.
\end{itemize}

\vspace{.3cm}

\Large
\textbf{Circle Correspondence}
\normalsize

We define $S_K$ as the Schmidt Arrangement of $\mathcal{O}_K$, where $K = \abs{\sqrt{\Delta}}$. Assume that $\Delta$ is a negative integer and that $0$ divides itself. 

We have a triple $(r,x,y) \in \mathbb{Z}^3$, which must adhere to 
\begin{equation}
r~|~x^2 + y + y^2.
\end{equation}

The triple results in a circle with curvature $2ir$, and a curvature-center $2(x+iy) + i$. We can represent the circle in $S_K$ by the matrix
\begin{equation}
\begin{pmatrix}
a + a'i & c + c'i \\
b + b'i & d + d'i
\end{pmatrix}
\end{equation}
which results in the the triple
\begin{equation*}
(m,n,l) =
\begin{cases}
(bd' - b'd, bc' - a'd, a'd' - b'c'), & \text{if}~\Delta = -1 \\
(b'd - bd', a'd - bc', bc - ad), & \text{otherwise}
\end{cases}
\end{equation*}

\begin{itemize}
	\item[\textbf{Case 1.}] $r = 0$. To find the corresponding circle in $S_K$ we let $a' = -\text{gcd}(x,y)$. Then
	\begin{equation*}
	a = \frac{-a'(1 + \norm{i}y)}{x}, ~~ c = c' = 0, ~~ b = d = \frac{-x}{a}, ~~ b' = d' = \frac{y}{a'}.
	\end{equation*}
	\begin{itemize}
		\item If $x=0$ and $y=0$ then $(r,x,y) = (0,0,0)$. This yields a $2 \times 2$ identity matrix.
		\item (IGNORE for now) If $x \neq 0$ then we find $a',a,b',b,c',c,d',d$ from the equations above, and where (1) is satisfied.
	\end{itemize}
	\item[\textbf{Case 2.}] $r \neq 0$. Let $p$ be a prime that divides $r$, $p~|~r$, and let $e$ be the exponent of the largest power of $p$ dividing a particular variable.
	
	We define $e_{b'}$, $e_d$, and $e_{d'}$ as 
	\begin{equation}
	e_d =
	\begin{cases}
	0 & e_{1+y} = 0 \\
	\text{min}(e_r, e_x) & \text{otherwise}
	\end{cases}
	~~~~\text{and}~~~~
	e_{b'}, e_{d'} = 
	\begin{cases}
	0 & e_y = 0 \\
	\text{min}(e_r, e_x) & \text{otherwise} 
	\end{cases}
	\end{equation}
	To find $d'$ we have that
	\begin{equation}
	d' = \prod_{p|r} p^{e_{d'}}.
	\end{equation}
	Then the following systems of congruences can be solved
	\begin{equation}
	dy \equiv -d'x~\text{mod}~p^{e_r}~~~~~~~~~~~dx \equiv d'(1 + y)~\text{mod}~p^{e_r},
	\end{equation}
	and
	\begin{equation}
	b'(dy + d'x) \equiv -ry~\text{mod}~p^{e_r + e_{d'}}~~~~~~~~~~~b'\left(dx - d'(1 + y)\right) \equiv -rx~\text{mod}~p^{e_r + e_{d'}}.
	\end{equation}
	
	We can then find the following variables
	\begin{align*}
	b &= \frac{r + b'd}{d'} & a &= \frac{bx - b'(1+y)}{r} \\
	a' &= \frac{by + b'x}{r} & c &= \frac{dx - d'(1+y)}{r} \\
	c' &= \frac{d'x + dy}{r}.
	\end{align*}
	\begin{itemize}
		\item \textbf{Example 1:} Let $(r,x,y) = (3,2,1)$ with $p=3$. Then from (3), $e_d = 0$, $e_{b'} = 0$, $e_{d'} = 0$. We also have that $e_r = 1$, since the largest power $p~|~r$ is 1. Then, by (4),
		\begin{equation*}
		d' = \prod_{1} p^{e_{d'}} = \prod_{1} 3^0 = \prod_{1} 1 = 1.
		\end{equation*}
		By (5),
		\begin{align*}
		& d \equiv -2 \bmod 3 \\
		& 2d \equiv -6 \bmod 3,
		\end{align*}
		which $d=1$ satisfies. By (6),
		\begin{align*}
		& 3b' \equiv -3 \bmod 3 \\
		& 0 \equiv -6 \bmod 3.
		\end{align*}
		which yields no solution for $b'$ (this is not a problem though). Solving for each variable
		\begin{align*}
		b &= 3 + b' & a &= \frac{2(3 + b') - 2b'}{3} = \frac{6}{3} = 2\\
		a' &= \frac{3 + b' + 2b'}{3} = 1 + b' & c &= 2 - 2 = 0 \\
		c' &= \frac{2 + 1}{3} = 1.
		\end{align*}
		Substituting into the matrix from (2),
		\begin{equation*}
		\begin{pmatrix}
		2 + i + ib' & i \\
		3 + b' + ib' & 1 + i
		\end{pmatrix}.
		\end{equation*}
		We can exclude $b'$ in the following way
		\begin{equation*}
		\begin{pmatrix}
		2 + i + ib' & i \\
		3 + b' + ib' & 1 + i
		\end{pmatrix}
		\cdot 
		\begin{pmatrix}
		1 & 0 \\
		-b' & 1
		\end{pmatrix}
		= 
		\begin{pmatrix}
		2 + i + ib' - ib' & i \\
		3 + b' + ib' - b' - ib' & 1 + i
		\end{pmatrix}
		=
		\begin{pmatrix}
		2 + i & i \\
		3 & 1 + i
		\end{pmatrix}
		\end{equation*}
		\item \textbf{Example 2:} Let $r,x,y) = (2,4,2)$ with $p = 2$. Then from (3), $e_d = 0$, $e_{b'} = 1$, $e_{d'} = 1$. We also have that $e_r = 1$, since the largest power of $p~|~r$ is 1. Then, by (4),
		\begin{equation*}
		d' = \prod_{2|2} p^{e_{d'}} = \prod_{1} 2^1 = \prod_{1} 2 = 2.
		\end{equation*}
		By (5),
		\begin{align*}
		& 2d \equiv -4 \bmod 2 \\
		&4d \equiv 6 \bmod 2,
		\end{align*}
		which $b' = 0$ satisfies. Solving for each variable
		\begin{align*}
		b &= \frac{2 + 0}{2} = 1 & a &= \frac{4 - 0}{2} = 2\\
		a' &= \frac{2 + 0}{2} = 1 & c &= \frac{0 - 6}{2} = -3 \\
		c' &= \frac{8 + 0}{2} = 4.
		\end{align*}
		Substituting into the matrix from (2),
		\begin{equation*}
		\begin{pmatrix}
		2 + i & -3 + 4i \\
		1 & 2i
		\end{pmatrix}.
		\end{equation*}
		\item \textbf{Example 3:} space for one more example, maybe another odd situation like example 1, if one is to arise.
	\end{itemize}
\end{itemize}

\clearpage

\Large
\textbf{Algorithm}
\normalsize

We present the following algorithm used in our program.

\begin{algorithm}
    \caption{: Check condition (1) and the $p~|~r$.}
    \label{euclid}
    \begin{algorithmic}[1] % The number tells where the line numbering should start
    \If {$r~|~x^2 + y + y^2$ is not True}
    	\State Raise an exception.
    \EndIf
    \If {($p$ is prime) is True}
    	\If {$p~|~r$ is not True}
		\State Raise an exception.
	\EndIf
    \Else
    	\State Raise an exception.
    \EndIf
    \end{algorithmic}
\end{algorithm}

\begin{algorithm}
    \caption{: Find $e_{d}$.}
    \textbf{Require:} Function 1 to hold true.
    \begin{algorithmic}[1] % The number tells where the line numbering should start
    \If {$p~|~(1 + y) = 0$}
    	\State $e_d \leftarrow 0$
    \Else
    	\State $e_r \leftarrow r/p$ \Comment{Want number of times $p~|~r$ and $p~|~x$}
	\State $e_x \leftarrow x/p$
	\State $e_d = \text{min}(e_r, e_x)$
    \EndIf
    \end{algorithmic}
\end{algorithm}





















\end{document}